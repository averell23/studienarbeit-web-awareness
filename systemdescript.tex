\documentclass[a4paper]{danarticle}
\usepackage{a4}
\pagestyle{empty}
\sloppy

\begin{document}
  \section*{The Test Platform (Gilbert)}
    \subsection*{Description}
      To test different methods of URL extraction, I developed an extensible
      framework that would allow the easy addition and removal of processing
      steps. Most of the actual implementation was done in Java, but since
      all cummunication is done through XML, modules can be written in any
      programming language.
    \subsection*{Processing Steps}
      FIXME: Enter processing diagram here \\
      \begin{enumerate}
        \item{The original log file data is converted into the LOG\_XML format.}
        \item{The \textbf{extraction} converts the LOG\_XML data into an url list
	      in the URL\_XML format.}
        \item{After the extraction, there may be a chain of \textbf{refiners}. 
	      Each refiner may modify the URL\_XML list in any way. Since the 
	      refiners are in a chain, the output of one refiner will be the 
	      input for the next one.}
        \item{The \textbf{presentation} is not defined by the framework. It will
	      usually consist of compiling the URL\_XML output of the last
	      refiner into a suitable display format.}
      \end{enumerate}
      In addition to this the Java implementation allows each extractor and
      refiner to contain \textbf{filters}. Filters are meant to discard 
      unwanted elements from the input before processing takes place.
      FIXME: ADD filter diagram.\\
      FIXME: ADD LOG\_XML example\\
      FIXME: ADD URL\_XML example\\
    \subsection*{Extractor Designs}
      \subsubsection*{Straight Extraction}
        This extraction method works on visits originating from IP addresses. 
	The extractor will try to resolve the address into a valid DNS hostname
	and, if successful, perform the following steps:
	\begin{enumerate}
	  \item{Try if the hostname is alive.}
	  \item{Iterate through all subdomains contained in the hostname, and
	        try if the address www.\{subdomain\} is alive.}
          \item{If an address is found to be alive, the url is written to the
	        output.}
	\end{enumerate}
	(In this context an address is said to be \textit{alive} if there is a
	service on the standard http port that responds to GET or HEAD requests.
	FIXME: ADD Example \\
    \subsection*{Refiner Designs}
      \subsubsection*{Search Engine Refinement}
        This refiner will use a standard web search engine (such as AltaVista
	or Google) to find more urls. The extractor uses a predefined set
	of keywords which are deemed to be interesting for the user.
	The extractor will take an url from the input and will search this
	url's domain for the given keywords. The results of this
	search are then written to the output.
      \subsubsection*{Meta Refinement}
        This refiner enriches the existing url information to faciliate the
	comparison and ranking of the results. The refiner will try to establish
	a http connection to a url and if successful will read the 
	<title> and some <meta> tags (like keywords) from the response. This
	information will then be written to the output. 
\end{document}
