\documentclass[a4paper]{danarticle}
\usepackage{a4}
\usepackage[dvips]{color}
\usepackage[dvips]{graphicx}
\pagestyle{headings}
\sloppy

\begin{document}
  \author{Daniel Hahn}
  \title{Visitor-Awareness Testbed (Gilbert)}
  \maketitle
  
  \section*{Introduction}
    Gilbert is a implementation of a Visitor-Related URL finding system. It
    consists of an XML data exchange format and Java classes containg the
    actual implementation.
    
    To understand what this software does one should refer to the
    description of the Visitor-Related URL finding systems - this
    document will refer to concepts introduced there.
  \section*{Components of Gilbert}
    Gilbert consists of two main components: \textit{Extractors} and 
    \textit{Refiners}.
    \\
    An \textit{Extractor} will typically take a list of visits and produce
    a list of URLs as an output. These URLs will then be starting points in
    the search for relations. An \textit{Extractor} is expected to perform
    the first step in the relation finding process: To find a URL close to
    a visit's origin. A well-behaved extractor should therefore return
    one URL for each visit on the input\footnote{This property is not enforced
    by the class, and is might be relaxed if neccessary: Duplicate URLs on
    the output, for example, can be eliminated.}.
    \\
    A \textit{Refiner} will typically take a list of URLs as an input and
    produce a list URLs as the output. It is expected that the output contains
    the URLs from the input and (optionally)
    any number of URLs related to them in the
    $ 1^{st} $ degree\footnote{This property isn't enforced either, and it
    may be relaxed to emulate filtering (so that URL that would have been
    dropped by a filter can be dropped in the \textit{Refiner} itself.}.
    The output of one \textit{Refiner} may be fed into another \textit{Refiner},
    creating a daisy-chain like structure. 
    //FIXME: Diagram needed
    
    A \textit{Refiner} may either search for URLs that are related to the
    input URLs, or may apply a \textit{interest function} to the Input URLs,
    thus performing the second and third step of the URL-finding process.
  \section*{Data flow/Degrees of Relationship}
    Since the \textit{Refiners} are configured in a daisy-chain configuration
    (as opposed to the more flexible configuration proposed for the theoretical
    system) a mechanism is needed to deal with different degrees of
    relationship: Each URL will carry a <degree> tag, indicating which degree
    of relationship the URL has to the starting URL. The \textit{Refiner}
    may opt to disregard all URLs which's <degree> is higher than a 
    predefined value\footnote{These URLs should still be passed through to
    the output, however.}. The degree of all new URLs in the output should be
    increased by one (depending on the degree of the URL which they are related
    to).
  \section*{Filtering}
    \textit{Extractors} and \textit{Refiners} may have \textit{prefilters}. 
    Filters registered with them will eiter \textit{accept} input objects, or
    not. Objects which are not accepted will never reach the \textit{Extractor}
    (or \textit{Refiner}) and will be lost for all further processing. Filters
    are generic, they can be used with all types of \textit{Extractors}  
    (or \textit{Refiners}, depending on the type of the filter).
    
  \section*{The Utility class}
    The utility class \textbf{gilbert.Util} contains methods for different
    recurring operations. It is recommended that the status of URLs is checked
    through the methods provided in this class, since they will be cached
    to speed up further processing.
    
  \section*{References for programmers}
    This document is only meant as a brief introduction, not as an up-to date
    documentation for programmers. The API documentation for the Gilbert classes 
    can be created with \textbf{javadoc}, it is also planned to provide 
    XML schemas as a reference for the XML data exchange 
    formats\footnote{It is recommended to run gilbert without document
    validation to speed up the processing.}.
\end{document}
