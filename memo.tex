\documentclass[a4paper]{danarticle}
\usepackage{a4}
\pagestyle{headings}
\sloppy
 
 
\begin{document}
  \author{Daniel Hahn}
  \title{Memo: Things about visitors}
  \maketitle
 
  \begin{abstract}
     These is a memo to structure some ideas, and it's a work in 
     progress. \textbf{Comments are always welcome:} dhahn@gmx.de
  \end{abstract}
 
  \section*{What it is about}
    The whole topic is quite broad, but loosely the following stuff needs to be
    done:
    \begin{itemize}
      \item{find out \textit{things} about the visitors to our site.}
      \item{filter out the \textit{things} needed.}
      \item{present the the \textit{things} to the user in a friendly way.}
    \end{itemize} 
    Sounds vague? Don't know what a \textit{thing} is? Well, at the moment it
    could be almost anything: Information about the user himself\footnote{No pun
    intended to the ladies, but in this case I'm writing the text as if the
    visitor is a male geek. She could also be a female geek, though.}, the user's
    homepage or other places on the web the user has connections with.
    The \textit{things} we are looking for will depend on what the system
    should achieve. Some possibilities include:
    \begin{itemize}
       \item{Just making the host aware of the visitors, or the number of
             visitors.}
       \item{Making the host aware of patterns in the visitor's behaviour.}
       \item{Actually pointing the host to a single visitor, getting as close
             as possible.}
       \item{Use information learned from the visitor to find interesting
             places on the web.}
       \item{Try provide an entry into a community that both the host and
             the visitor are part of.}
    \end{itemize}
    All theses things are probably interesting to explore, but the scope of a
    Studienarbeit will probably only allow to implement one of them.
  \section*{Find things}
    There are several sources of information that could be tapped to learn more
    about the visitor. For now we assume from all these sources we can learn
    something about \textit{visits}, where a \textit{visit} is the action of
    actually looking at a web page. (A line in a log file would correspond to 
    a \textit{visit}).
    \subsection*{Logfiles}
      Web server logfiles contain a record for each request that has been
      served. In the best case, there is information about the \textbf{hostname}
      or \textbf{ip address} the request came from, a \textbf{timestamp}, a
      \textbf{status} indication (if the transaction was successful or not),
      information about the \textbf{user agent} that made the request and also
      about the \textbf{referring page}, which sent the user to this server.
      
      If a users logs into a site using built-in security mechanisms, the
      respective \textbf{username} could also be made available.
      
      In addition, by looking at the whole logfile we could also answer
      questions like: \textit{How often does a visitor come back?} or \textit{What
      type of visitor is this?}
    \subsection*{Transfer of additional information(through protocol extensions)}
      Although there is a lot of data available in the logfile, it is not meant
      to lead to interesting places. An obvious solution would be to extend the
      http protocol, to allow the visitor (or rather, his user agent) to provide
      additional information. Interesting information could include: The
      visitor's \textbf{most frequently visited site}, his \textbf{bookmark
      collection} or simply his \textbf{own homepage}. 
      
      The major setback (apart from privacy considerations) of this approach is
      that it depends on modifications both in the server and the user agent.
    \subsection*{User provided information} 
      Users often provide useful information themselves: By writing
      \textbf{guestbook} entries, by \textbf{signing up} for services or by completing
      \textbf{surveys}.
    \subsection*{Secondary sources - the web and more}
      Once we learned some information about a visit, we can always try to look
      up more interesting facts. Of course a \textbf{WWW search} is the most
      obvious example, but we could also use things like \textbf{local
      databases}, \textbf{traceroute information} or \textbf{WHOIS lookups} to
      name but a few.
      
      Another source of secondary information may be the \textbf{host himself}
      who can tell the system which results he liked best (or worst).
  \section*{Select the things to play with}
    Once we have gathered information about a lot of visits we will find that
    most of it is probably not very interesting in a given context. For example,
    although a \textit{visit} by a search robot is an interesting thing for some
    webmasters, the same \textit{visit} is quite uninteresting if you are looking
    for visitors with the same interests than the host.\footnote{Unless the host
    is interested in indexing web pages all day, of course...}
    \subsection*{Look at the things you got}
      The first step could be a classification of each \textit{visit}. Taking into
      account the information learned both from this visit and previous visits
      we could start asking questions like: \textbf{What kind of visitor is
      this?} \textbf{What is he looking at?} or \textbf{Did he come from a
      university?}
    \subsection*{Throw away the unwanted things}
      Once we have learned a lot of \textit{things} about our visitors, 
      we are likely to
      throw out the things we don't need. For example, we could now eliminate
      visitors which are \textbf{searchbots}, come from a certain
      \textbf{service provider} and so on.
    \subsection*{Rank the things remaining}
      Once we have a set of \textit{things} that is potentially interesting, 
      we want to
      rank them (the most interesting first...) Indications for the quality of a
      thing could include: The \textbf{type of the visitor}, the \textbf{number
      of visits} by that visitor, \textbf{similiarity} to our own page, and many
      more...
  \section*{Dish out the good things}
    Once we have selected the most interesting \textit{things} for our 
    purpose (whatever
    that may be), we are ready to present them to the interested public.
    \subsection*{Web page}
      The simplest approach to present the results. This could be a page that
      the host explicitly calls to learn about the visitors. This could also be
      a small frame/overlay in an existing page that feeds the results back to
      visitors and hosts alike.
    \subsection*{Ambient display/Installation}
      A gentler method of making the host aware of the visitors could be an
      ambient display. This could be as simple as a light that grows brighter
      when more visitors come, or very sophisticated as an "artwork" that
      actually conveys a lot of information.
    \subsection*{Community display}
      This could be a display that stays in the background (like an ambient
      display) but uses a lower level of abstraction and allows the user to
      actually surf into the pages if he chooses to do so.
    \subsection*{Personal display}
      This would probably consist of something as a web pad, working much like
      the community display does but intended to be used rather like a
      newspaper.
\end{document}
