\documentclass[a4paper]{danarticle}
\usepackage{a4}
\pagestyle{headings}
\sloppy

\begin{document}
  \section{Visitor awareness on the Web: Studienarbeit Daniel Hahn}
    This should be a short exploration on what topics should be included in the
    final work, and what the results should be
    \subsection{Visitors on the web}
      The situation is the following: We have some information about the
      visitors/visits to a web site (we could even gather more). We want to make
      the host aware of her visitors, and find out if there are interesting
      places on the web which the visitors are connected with. This takes up the
      topic of a recent paper of Gellersen/Schmidt\footnote{Complete details
      for paper still missing here...} an tries to expand it.
    \subsection{Parts of the system}
      To do all this, I want to create a system which performs basically the
      following steps: Gather information about visitors. Use this information
      to find places on the web (e.g. URLs). Find out which of these are most
      interesting to the user. Present these to the user, making her aware of
      what's going on.
      \subsection{Topics on information gathering}
        For a start, gathering information about visitors means parsing and
	understanding log files and I may do some basic classification (type of
	visitor, location, etc.). Further research could go into finding other
	ways of gathering information, e.g. having user agents that
	transmit interesting URLs during requests. Also, the existing
	information could be analyzed to find more metainformation (like: "`this
	is a regular visitor"')
      \subsection{Topics on finding places on the Web}
        One can imagine different techniques to extract interesting URLs from
	the raw data: From simple mechanisms like looking up hosts in the
	visitor's domain to sophisticated search algorithms and the use of the
	visitor's search keywords and links, everything is possible.
	My aim is to develop such methods, to analyse them and to evaluate their
	quality in terms of how "`interesting"' the results are (and also, under
	which conditions they will perform best).
      \subsection{Topics on measuring how interesting places are}
        Once URLs are found, one has to decide which are likely to be the most
	interesting for the host to look at. This could be done by simply using
	statistical methods (e.g. showing the page that came up most often),
	but more sophisticated methods are also possible: One could look for
	certain keywords or for simililarities with other interesting pages.
	My objective here would be to develop such techniques and do a
	preliminary analysis of their performance (because the
	"`interestingness"' can of course only be measured through an actual
	user study).
      \subsection*{Topics for presenting the data}
        To present the data, I would need to implement a device that actually
	allows the host to make use of the information. This could, for example,
	be a web pad that shows links to various sites on the net.
    \subsection{Testbed implementation}
      I also aim to come up with a sensible system architecture and a testbed
      implementation to analyse the various methods. The implementation will
      probably rather simple to begin with, but should evolve during the
      process. 
    \subsection{Further topics}
      There are other interesting research topics connected with this area. One
      would be an user study to determine the actual usefulness of the system,
      another an analysis of the privacy issues.
    \subsection{Closing note}
      The topics 1.4 and 1.5 offer probably the most interesting research
      possibities and will probably be looked into more detailled than the other
      topics.
\end{document}


