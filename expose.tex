\documentclass[a4paper]{danarticle}
\usepackage{a4}
\pagestyle{empty}
\sloppy

\begin{document}
  \author{Daniel Hahn}
  \title{Web Guests\\ Proposal for a minor thesis project\\ (Studienarbeit)}
  \maketitle
    The perception of the World Wide Web is shifting from that of a mere system
    for information display and retrieval to one of a social place, consisting
    of virtual \lq\lq places\rq\rq , where people \lq\lq go\rq\rq\ and \lq\lq
    meet\rq\rq . It isn't surprising to find that interaction between the denizens
    of the net -- and especially finding new metaphors for their interaction --
    is a fascinating research topic.
    
    In recent publications, Gellersen and Schmidt introduced the concepts of
    \textit{visitor awareness} and \textit{glances into visitor's sites}
    \footnote{e.g.: \lq\lq Look who's visiting: supporting visitor awareness in 
    the web, IJHC}. 
    The aim of \textit{glances into visitor's sites} 
    was to \lq\lq visit the visitor back\rq\rq\ by looking up
    information in the host's log file and trying to find the visitor's own
    home page. The system had a dedicated display that showed the pages of the
    current visitors, and the host had the option to browse further if she liked
    to.
    
    First studies of this original system showed that the hosts were pleased to
    have this information presented to them. It also showed that it may
    successfully be used in building virtual communities.
    
    The objective of this project is to explore the possibilities of a system
    using the \textit{glances into visitor's sites} approach. We will focus
    mainly on the different methods that can be used to analyse the visitor
    information (log files, in this case), to find interesting home pages (or
    \lq\lq web places\rq\rq ) and to distinguish interesting pages from
    uninteresting ones. 
    \\
    
    \textbf{1. Analysis of Design Space.} Since a system like this
    has not been designed before, it will be important to 
    develop a concept of what it's possibilities and limitations are.
    
    In this step we will try to identify the different components that are
    needed to analyse the visitor information and to find (and  also rate) the
    URLs. It is likely that for each of the components there are many different
    methods to approach it's task. We will need to come up with a number of such
    methods and evaluate which ones are worth to be tried out.
    \\
    At the end of this phase we should have fairly good idea of what the system
    will look like: What components are needed, how they should interact and
    what approaches (or algorithms) can be used for each of them.
    \\
    
    \textbf{2. Create building blocks.} In a next step, we will create
    different building blocks to find out how our theoretical 
    concepts perform in a real-world setting. Each of these blocks  will
    be a re-usable component and implement one particular processing method.
    
    At the end of this phase, we will have not only the components to build a
    complete system, but also an analysis on  how the different 
    components (and the methods used in them) perform under different
    circumstances.
    \\
    
    \textbf{3. Sample Implementation.} As soon as we have the necessary
    building blocks, they can be used them to build a complete implementation of the
    system. Such a sample implementation will also have to include some form of
    output (e.g. a wall screen) and will serve different purposes: For one part,
    we will be able to see how our components work together in a complete
    system and allow us to experiment with the parameters. We will also be able
    to get a first impression of the actual usefulness of the system, and have a
    starting point for a proper user study (which is outside the scope of this
    project).
    \\
    
    \textbf{4. Other issues}
    In addition to the tasks mentioned above, the project should also take
    at least a glance at the environment into which such a system may be
    deployed. This could possibly include legal and privacy issues, but 
    also reflections on what impact this may have on our perception of 
    the web.
\end{document}
