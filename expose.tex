\documentclass[a4paper]{danarticle}
\usepackage{a4}
\pagestyle{headings}
\sloppy

\begin{document}
  \section*{More Visitor Awareness}
    The perception of the World Wide Web is shifting from that of a mere system
    for information display and retrieval to one of a social place, consisting
    of virtual \lq\lq places\rq\rq , where people \lq\lq go\rq\rq\ and \lq\lq
    meet\rq\rq . Intraction between the denizens of the net has always been an
    interesting research topic - the challenge is not to simulate the enviroment
    of real-world-encounters, but find ways of interaction that are adequate for
    the new medium.
    
    In recent publications, Gellersen and Schmidt introduced the concepts of
    \textit{visitor awarenes} and \textit{glances into visitor's sites}
    \footnote{Quote the correct paper... yada yada...}. While their work
    focussed on making the host aware of the general traffic at his site (e.g.
    by turning appliances on and of, or by changing the brightness of a light),
    the \textit{glances into visitor's sites} followed a different approach.
    Here the aim was to \lq\lq visit the vistor back\rq\rq\ by looking up
    information in the host's logfile and trying to find the visitor's own
    homepage.
    
    Coming from the \textit{glances into visitor's sites} approach, it will be
    interesting to research further approaches to this kind of visitor
    awareness. Showing the visitor's web site is just one method of making the
    host aware of what's going on, and in theory the possibilities are as unlimited
    as the forms of everyday interaction are: In real life, visitors can stay
    silent, make themselves known, ask questions or or even offer helpful
    advice. Also, in more private settings, having visitors is a self-serving
    activity, and only undertaken for the pleasure of the social setting.
    
    The objective of this project is to explore the possibilities of a system 
    that uses visitor information to make the host aware of places she is
    (indirectly) connected to. The system will analyze the existing information
    and compile it into a useful respresentation of the host's virtual 
    \lq\lq neighbourhood\rq\rq\footnote{Actually, this term is a little 
    misleading in this context, but I haven't found similiarly suitable 
    word for \lq\lq The Coming and Going of Virtual Visitors\rq\rq\ yet.}
    
    One major task here is the analysis of the design space: Since such 
    a system has not been designed before, it will be important to 
    develop a concept of what it's possibilities and limitations are.
    At first we need to examine what the available visitor information 
    is and by which means additional information can be made available.
    We also have to look at the user's end, finding out which type of 
    information the user of our system needs and how it relates to the
    available data. The final result of this step is twofold: First, we
    should have a clear idea of what our system is able to achive with
    different types of input data and what it's possible applications are.
    Second, we should have established the processing steps that are
    neccessary to go from the original visitor information to the whatever
    output we desire for the application at hand. We will also need to 
    separate the global processing steps (that need to be performed
    for all applications) from the different processing methods that will
    vary depending on the purpose of the system.
    
    At the same time, we need to find out which of the theoretical ideas
    are actually workable, thus we will build a testbed to implement 
    different building blocks for the system. This will allow us to 
    examine different processing methods and also improve the theoretical
    model of our system. The outcome of this phase should be an analysis
    of different processing methods; we will then have a better
    understanding of how a real-world system may look like and how our
    methods perform under real-worl conditions.
    
    At this point, we can build a sample implementation geared towards 
    one of the applications we have previously identified. (This could,
    for example, be a communal display with the visitor's web pages)
    Such a system will allow us to experiment with different parameters,
    getting a first feedback on how it performs. It would also allow for
    a more thourough user study, which would be neccessary to determine
    the actual usefulness of it (although such a thing is time-consuming
    and may be beyond the scope of the current project).
    
    In addition to the tasks mentioned above, the project should also take
    at least a glance at the enviroment into which such a system may be
    deployed. This could possibly include legal and privacy issues, but 
    also reflections on what impact this may have on our perception of 
    the web.
\end{document}
