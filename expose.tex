\documentclass[a4paper]{danarticle}
\usepackage{a4}
\pagestyle{empty}
\sloppy

\begin{document}
  \author{Daniel Hahn}
  \title{Web Guests\\ Proposal for a minor thesis project\\ (Studienarbeit)}
  \maketitle
  \section*{More Visitor Awareness}
    The perception of the World Wide Web is shifting from that of a mere system
    for information display and retrieval to one of a social place, consisting
    of virtual \lq\lq places\rq\rq , where people \lq\lq go\rq\rq\ and \lq\lq
    meet\rq\rq . It isn't surprising to find that interaction between the denizens
    of the net -- and especially finding new metaphors for their interaction --
    is a very interesting research topic.
    
    In recent publications, Gellersen and Schmidt introduced the concepts of
    \textit{visitor awarenes} and \textit{glances into visitor's sites}
    \footnote{Quote the correct paper... yada yada...}. While their work
    focussed on making the host aware of the general traffic at his site (e.g.
    by turning appliances on and of, or by changing the brightness of a light),
    the \textit{glances into visitor's sites} followed a different approach.
    Here the aim was to \lq\lq visit the vistor back\rq\rq\ by looking up
    information in the host's logfile and trying to find the visitor's own
    homepage.
    \\
    Taking this approach a step further is certainly an interesting research
    oppurtunity. Not only is there a number of approaches to finding and
    analyzing visitor's pages, but we may also learn more about the
    possibilities of host/visitor interaction. At least in theory, the
    possibilities are as unlimited as in the real world: Visitors may come for
    information, for business or just for the pleasure of the social setting
    itself. 
    
    The objective of this project is to explore the possibilities of a system 
    that uses visitor information to make the host aware of places she is
    (indirectly) connected to. This project will focus on a system that 
    uses visitor
    information which is already available (e.g. web server logfiles) and
    compiles it into a list of \lq\lq places\rq\rq\ (here: URLs) that may be
    interesting for the host to visit. This URLs may then presented in any
    appropriate way to make the host aware of her virtual \lq\lq
    surroundings\rq\rq . 
    \\
    
    \textbf{1. Analysis of Design Space.} Since a system like this
    has not been designed before, it will be important to 
    develop a concept of what it's possibilities and limitations are.
    
    In this step we will try to identify the different components that are
    needed to analyze the visitor information and to find (and  also rank) the
    URLs. It is likely that for each of the components there are many different
    methods to approach it's task. We will need to come up with a number of such
    methods and evaluate --  based on current knowledge -- 
    which of these may be most useful.
    \\
    At the end of this phase we should have fairly good idea of what the system
    will look like: What components are needed, how they should interact and
    what approaches (or algorithms) can be used for each of them.
    \\
    
    \textbf{2. Create building blocks.} In a next step, we will create
    different build blocks to find out how our theoretical 
    concepts perform in a real-world setting. Each of our building blocks  will
    be a re-usable component and will implement one of the methods we designed.
    
    At the end of this phase, we will have not only the components to build a
    complete system, but will have an analysis on  how the different 
    components (and the methods used in them) perform under different
    circumstances.
    \\
    
    \textbf{3. Sample Implementation.} As soon as we have the neccessary
    building blocks, we can use them to build a complete implementation of the
    system. Such a sample implementation will also have to include some form of
    output (e.g. a wallscreen) and will serve different purposes: For one part,
    we will be able to see how our components work together in a complete
    system and allow us to experiment with the parameters. We will also be able
    to get a first impression of the actual usefulness of the system, and have a
    starting point for a proper user study (which is outside the scope of this
    project).
    \\
    
    \textbf{4. Other issues}
    In addition to the tasks mentioned above, the project should also take
    at least a glance at the enviroment into which such a system may be
    deployed. This could possibly include legal and privacy issues, but 
    also reflections on what impact this may have on our perception of 
    the web.
\end{document}
